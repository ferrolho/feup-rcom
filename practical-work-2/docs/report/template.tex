\documentclass[a4paper]{article}
\usepackage[portuguese]{babel}
\usepackage[utf8]{inputenc}
\usepackage{indentfirst}
\usepackage{graphicx}
\usepackage{verbatim}


\begin{document}

\setlength{\textwidth}{16cm}
\setlength{\textheight}{22cm}

\title{\Huge\textbf{Trabalho 2}\linebreak\linebreak\linebreak
\LARGE{Configuração de uma rede e desenvolvimento de uma aplicação de download}\linebreak\linebreak
\Large\textbf{Relatório Final}\linebreak\linebreak
\includegraphics[scale=0.1]{feup-logo.png}\linebreak\linebreak\linebreak
\Large{Mestrado Integrado em Engenharia Informática e Computação} \linebreak\linebreak
\Large{Redes de Computadores}
}
\author{\textbf{Professor:}\\ Manuel Ricardo\\\\\textbf{Turma 4:}\\ Henrique Manuel Martins Ferrolho - ei12079 \\ João Filipe Figueiredo Pereira - ei12023 \\ José Pedro Vieira de Carvalho Pinto - ei12164 \\ Miguel Ângelo Jesus Vidal Ribeiro - ei11144\\\linebreak\linebreak \\
 \\ Faculdade de Engenharia da Universidade do Porto \\ Rua Roberto Frias, s\/n, 4200-465 Porto, Portugal \linebreak\linebreak\linebreak
\linebreak\linebreak\vspace{1cm}}
\maketitle
\thispagestyle{empty}

%************************************************************************************************
%************************************************************************************************

\newpage

\section*{Resumo}
Por fazer.

\newpage

\tableofcontents
\newpage

\section{Introdução}
Por fazer.

\section{Parte 1 - Aplicação de download}
Por fazer.

\subsection{Arquitetura}
Por fazer.

\subsection{Resultados de download}
Por fazer.

\section{Parte 2 - Configuração da rede e análise}
Network architecture, experiment objectives, main configuration commands
Analysis of the logs captured that are relevant for the learning objectives
Por fazer.

\subsection{Experiência 1 - Configurar um IP de rede}
Por fazer.

\subsection{Experiência 2 - Implementar duas LAN's virtuais no switch}
Por fazer.

\subsection{Experiência 3 - Configurar um router em Linux}
Por fazer.

\subsection{Experiência 4 - Configurar um router comercial e implementar o NAT}
Por fazer.

\subsection{Experiência 5 - DNS}
Por fazer.

\subsection{Experiência 6 - Ligações TCP}
Por fazer.

\section{Conclusões}
Por fazer.

\clearpage
\addcontentsline{toc}{section}{Referências}
\renewcommand\refname{Referências}
\bibliographystyle{plain}
\bibliography{myrefs}


\newpage
\appendix
\section{Anexos}

\subsection{Código da aplicação}
Código da aplicação.

\subsection{Comandos de configuração}
Comandos de configuração.

\subsection{Logs gravados}
Logs gravados.

\end{document}
